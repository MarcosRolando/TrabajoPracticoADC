\documentclass[11pt,a4paper]{report}
\usepackage[document]{ragged2e}
\usepackage{setspace}
\usepackage[utf8]{inputenc}
\usepackage{amsmath}
\usepackage{amsfonts}
\usepackage{amssymb}
\usepackage{textcomp}
\usepackage{gensymb}
\usepackage[font=small,labelfont=bf]{caption}
\usepackage{graphicx}
\graphicspath{ {~/LatexFiles/TrabajoPracticoADC} }
\author{Marcos Rolando}
\begin{document}
\spacing{1.25}

\section*{Análisis general de la función de transferencia}

\subsection*{Tipo de filtro}

Dada la función de transferencia asignada
\[H(s)=\frac{3948 \cdot s^2}{s^4+88,86 \cdot s^3+7,935 \cdot 10^5 \cdot s^2+3,508 \cdot 10^7 \cdot s+1,559 \cdot 10^{11}}\]
procederemos a analizar su comportamiento en $s=0$ y $s\longrightarrow\infty$.

\bigskip
En primer lugar analizamos el caso $s=0$ para el cual tenemos que 
\[H(0) = \frac{3948 \cdot 0^2}{0^4+88,86 \cdot 0^3+7,935 \cdot 10^5 \cdot 0^2+3,508 \cdot 10^7 \cdot 0+1,559 \cdot 10^{11}}\]

obteniendo entonces que
\[H(0) = 0\]

Analizando ahora el caso $s\longrightarrow\infty$ tenemos que
\[\lim_{s \to \infty} H(s) = 0\]
dado que el grado del denominador es dos veces mayor al del numerador.

\bigskip
En base a los valores obtenidos podemos entonces afirmar que se trata de un
filtro pasabanda dado que la transferencia es nula para frecuencias bajas y altas.

\subsection*{Ceros y polos}

En principio, es trivial ver que el único cero de la función $H(s)$ es
$s = 0$ y dado que está elevado al cuadrado se deduce que el cero es doble.

\bigskip
Por otro lado tenemos los polos, cuyo cálculo no es trivial. Dado que el denominador de la función es de grado cuatro tendremos entonces cuatro raíces. Debido a esto
y a que los coeficientes del polinomio complejizan el desarrollo del cálculo de 
dichas raíces, se calcularán entonces mediante calculadora. 

\bigskip
Obtenemos que
\[p_{1,2} \approx -21,40395436 \pm 606,8764861j\]
\[p_{3,4} \approx -23,02604564 \pm 649,8008994j\]
donde $p_{1,2}$ y $p_{3,4}$ son los pares conjuados que componen los cuatros
polos de $H(s)$, y donde el primer subíndice corresponde al conjugado cuya parte imaginaria es positiva mientras que el segundo subíndice corresponde al de parte imaginaria negativa.

\subsection*{Cálculo de $W_{0}$ y Q}

Dado que el denominador se compone de dos pares de raíces complejas conjugadas tendrá entonces un $W_{0}$ y Q para cada par. Para obtenerlos debemos primero reescribir la función $H(s)$ a la forma
\[H(s)=\frac{3948 \cdot s^2}{(a \cdot s^2+b \cdot s+c) \cdot (d \cdot s^2+e \cdot s+f)}\]
donde las letras corresponden a valores que debemos calcular de forma tal que dicha
ecuación sea equivalente a la fórmula original de $H(s)$.

\bigskip
Para conseguir esto es conveniente utilizar las raíces del denominador de $H(s)$ (los polos de la función) expresando la función como
\[H(s)=\frac{3948 \cdot s^2}{k \cdot (s-p_{1}) \cdot (s-p_{2}) \cdot (s-p_{3}) \cdot (s-p_{4})}\]
donde k es el factor del término $k*s^4$ y, para nuestra función en particular,
se da que $k=1$. 

\bigskip
Luego, multiplicando los polos conjuados entre sí obtenemos
\[H(s)=\frac{3948 \cdot s^2}{(s^2-(p1+p2) \cdot s+p_{1} \cdot p_{2}) \cdot (s^2-(p3+p4) \cdot s+p_{3} \cdot p_{4})}\]

\bigskip
Sean $z_{1}$ y $z_{2}$ dos números complejos conjugados tenemos que 

\[z1+z2=2\Re(z_{1}) = 2\Re(z_{2})\] 
\[z1 \cdot z2=|z_{1}|^2=|z_{2}|^2\] 

Aplicando estas propiedades a $p_{1,2}$ y 
$p_{3,4}$ obtenemos la expresión
\[H(s)=\frac{3948 \cdot s^2}{(s^2-2\Re(p1) \cdot s+|p_{1}|^2) \cdot (s^2-2\Re(p3) \cdot s+|p_{3}|^2)}\]

\bigskip
$W_{0}$ y Q vienen dados por $s^2+\frac{W_{0}}{Q} \cdot s + W_{0}^2$. Ejemplificando para el primer factor del denominador tendríamos entonces que 
\[W_{0_{1}} = |p1|\]
\[Q_{1} = -\frac{W_{0_{1}}}{2\Re(p_{1})}\]

\bigskip
Reemplazando obtenemos finalmente que
\[W_{0_{1}} \approx 607,2538173 \frac{r}{s}\]
\[Q_{1} \approx 14,18555205\]
\[W_{0_{2}} \approx 650,2087416 \frac{r}{s}\]
\[Q_{2} \approx 14,11898404\]
siendo $W_{0_{1}}$ y $Q_{1}$ los valores correspondientes al polinomio de segundo 
grado cuyas raíces son $p_{1,2}$ y $W_{0_{2}}$ y $Q_{2}$ los correspondientes
al polinomio de segundo grado de raíces $p_{3,4}$.

\bigskip
Finalmente, reemplazando con los valores de W y Q obtenidos podemos expresar la función de transferencia como
\[H(s) \approx \frac{3948 \cdot s^2}{(s^2+42,81 \cdot s + 607,3^2)
\cdot (s^2+46,05 \cdot s + 650,2^2)}\]

\subsection*{Diagramas de Bode}

A continuación se presentan los diagramas de Bode tanto de módulo como de fase (en grados sexagesimales) de la función $H(s)$ junto con una breve descripción explicando
lo obtenido. Para los diagramas de Bode se analiza el caso $s=jw$ donde $w$ se mide
en $\frac{r}{s}$ (radianes por segundo).

\begin{figure}[h!]
\includegraphics[scale=0.7]{DiagramasBode.png}
\caption{Diagramas de Bode}
\end{figure}

\newpage
En primer lugar tenemos el diagrama de módulo de Bode, es decir, el módulo de 
$H(jw)$ medido en dB (decibeles) el cual viene dado por la\\ fórmula
\[20 \cdot \log(|H(jw)|) dB\]
Dado que el filtro es un pasabanda era lo esperable
observar que para frecuencias bajas ($w\longrightarrow0$) y altas ($w\longrightarrow\infty$) el gráfico tendiera a $-\infty$ dado que este tipo de
filtro se caracteriza por anular la función de transferencia para dichas frecuencias.
Esto puede verse matemáticamente en base al análisis previo realizado donde se calculó que $H(0) = 0$ y $\lim_{s \to \infty} H(s) = 0$, luego 
$\lim_{x \to 0} \log(x) = -\infty$ lo cual explica lo observado en el primer gráfico.
Por último, para las frecuencias en el intervalo $(W_{0_{1}}, W_{0_{2}})$ se aprecia una ganancia que tiende a $0dB$ lo que implica una salida de igual amplitud a la de
la señal de entrada (esta es la banda que nuestro filtro deja pasar sin amortiguar).

\bigskip
El segundo gráfico es el diagrama de fase de Bode (en grados sexagesimales). 
Los valores de este gráfico vienen dados por la fórmula 
\[\arctan(\frac{\Im(H(jw))}{\Re(H(jw)})\]
Dado que el numerador de la función se compone de un único término $3948 \cdot s^2$ 
tenemos que la constante positiva aporta $0\degree$ mientras que el $s^2$ aporta
$180\degree$ (recordemos que $s=jw$ lo cual tiene un ángulo de $90\degree$, luego elevar al cuadrado duplica el ángulo y obtenemos $180\degree$).

\bigskip
Al acercarnos a las frecuencias $W_{0}$ de los polos vemos como empieza a disminuir el ángulo a un ritmo de $-180 \frac{grad}{dec}$ aproximadamente. En rigor, en principio hay un intervalo donde disminuye $-90 \frac{grad}{dec}$ pero dicho intervalo no es apreciable dada la escala y cercanía entre los $W_{0}$. 
Finalmente, una vez alcanzado el valor $W=10 \cdot W_{0}$ los polos ya no aportan
pendiente decreciente y se estabiliza el gráfico nuevamente, quedando en este caso
en $-180\degree$ aproximadamente.

\newpage
\section*{Respuesta gráfica del sistema a distintas señales}

A continuación se presentan gráficos de respuestas del sistema a distintos tipos
de señales y frecuencias.

\subsubsection*{Respuesta al escalón}

\begin{figure}[h!]
\includegraphics[scale=0.7]{RtaEscalon.png}
\caption{Respuesta al escalón}
\end{figure}

Vemos que al ser un filtro pasabanda nuestro sistema no reacciona al instante 
$t=0^{+}$ el cual correspondería a una frecuencia $w\longrightarrow \infty$.
Luego obtenemos una respuesta oscilatoria (esperable dado que se tienen polos
complejos) que va tendiendo a cero a medida que pasa el tiempo (debido a los factores
exponenciales decrecientes provenientes de la parte real negativa de las raíces de los polos). Cuando $t\longrightarrow \infty$ tenemos que $w\longrightarrow 0$ y
se da, como era esperable dado el análisis previo, que la respuesta del sistema tiende a cero y logrando estabilizarse.

\bigskip
Si analizamos el instante donde $t = 0,15s$ podemos observar un comportamiento en principio extraño, donde nuestra señal disminuye considerablemente para luego volver a crecer y finalmente anularse. Esto se debe a que la respuesta al escalón es una función compuesta por funciones senoidales de igual fase y frecuencias muy cercanas entre sí (recordemos que $W_{0_{1}} \approx 607,3\frac{r}{s}$ y $W_{0_{2}} \approx 650,2\frac{r}{s}$) donde la amplitud de estas senoidales viene dada por funciónes exponenciales negativas de $\tau$ (\textit{taus}) cercanos entre sí ($\tau_{1} \approx 21,03s$, $\tau_{2} \approx 23,03s$), lo que da como resultado el efecto de \textit{batido} para esa intervalo de tiempo específico. Los siguientes gráficos ayudan a ejemplificar este efecto de batido.

\bigskip
\begin{figure}[h!]
\centering
\includegraphics[scale=1]{EfectoBatido.png}
\caption{Efecto de batido entre dos funciones senoidales}
\end{figure}

Si no tuvieramos las exponenciales negativas que nos "matan" la señal observaríamos lo siguiente a medida que avanzacemos en la escala del tiempo.

\bigskip
\begin{figure}[h!]
\centering
\includegraphics[scale=1]{EfectoBatidoExtendido.png}
\caption{Efecto de batido extendido en el tiempo}
\end{figure}

\newpage
\subsubsection*{Respuesta al impulso}

\begin{figure}[h!]
\includegraphics[scale=0.7]{RtaImpulso.png}
\caption{Respuesta al impulso}
\end{figure}

La respuesta al impulso es matemáticamente la derivada de la respuesta al escalón,
lo que dará como resultado que en los valores de $t$ donde la respuesta al escalón
alcance un máximo o mínimo entonces en la respuesta al impulso obtendremos un cero.
Puede observarse sencillamente esto en el $t$ alineado con la barra horizontal 
del valor $0 V$, donde para la respuesta al escalón se ve que se alcanza un máximo
mientras que la respuesta al impulso vale cero.



\newpage
\subsubsection*{Respuesta a senoidales}

\begin{figure}[h!]
\includegraphics[scale=0.96]{RtaSenoidalBajo.png}
\caption{Respuesta a la senoidal de frecuencia $w = 1\frac{r}{s}$}
\end{figure}

Para una señal senoidal de frecuencia $w = 1\frac{r}{s}$ y $1V$ de amplitud vemos que la respuesta es prácticamente nula (tenemos una amplitud de salida de aproximadamente $1,3mV$, un $0,13\%$ de la señal de entrada) coincidiendo con el comportamiento esperado del filtro dado que es un pasabanda. 

\newpage
\begin{figure}[h!]
\includegraphics[scale=1]{RtaSenoidalMedio.png}
\caption{Respuesta a la senoidal de frecuencia $w = 630\frac{r}{s}$}
\end{figure}

Vemos que para una señal senoidal de frecuencia $w = 630\frac{r}{s}$ y amplitud
$1V$ al estabilizarse la respuesta obtenemos que la amplitud de la misma coincide aproximadamente con la de la señal senoidal. Esto coincide con el análisis
previo de la función de transferencia, ya que $w = 630\frac{r}{s}$ se encuentra a mitad de camino entre los $W_{0}$ calculados ($W_{0_{1}} \approx 607,3 \frac{r}{s}$ y 
$W_{0_{2}} \approx 650,2 \frac{r}{s}$) y vimos en el diagrama de Bode que para la banda de frecuencias con $W_{0_{i}}$ de extremos la ganancia era de aproximadamente $0dB$, lo que significa que la la salida mantiene la amplitud de la entrada, tal y como podemos apreciar para la respuesta a esta señal senoidal.

\newpage
\begin{figure}[h!]
\includegraphics[scale=0.98]{RtaSenoidalAlto.png}
\caption{Respuesta a la senoidal de frecuencia $w = 5000\frac{r}{s}$}
\end{figure}

Por último tenemos que para una señal senoidal de frecuencia $w = 5000\frac{r}{s}$ y amplitud $1V$ la respuesta de nuestro sistema es prácticamente nula al igual que para
el análisis de las bajas frecuencias (como vimos previamente para la señal senoidal de frecuencia $w = 1\frac{r}{s}$). La respuesta observable en el 
gráfico coincide, lógicamente, con el hecho de que nuestro filtro es un pasabanda y
por lo tanto anulará las señales de altas frecuencias.

\newpage
\subsubsection*{Respuesta a la cuadrada}

Esta sección se dividirá en dos subsecciones, en la primera analizaremos las cuadradas de frecuencias relacionadas con la de $W_{0_{1}} \approx 607,3 \frac{r}{s}$ y en la siguiente para las cuadradas de frecuencias relacionadas a $W_{0_{2}}$. Recordemos que $f = \frac{W}{2 \cdot \pi}$

\subsubsection*{Frecuencia $f_{0_{1}}$}

Para $W_{0_{1}}$ tenemos que $f_{0_{1}} = \frac{W_{0_{1}}}{2 \cdot \pi} \approx 96,65Hz$.

\begin{figure}[h!]
\includegraphics[scale=1]{RtaCuadradaWo11.png}
\caption{Respuesta a la cuadrada de frecuencia $\frac{f_{0_{1}}}{10}$}
\end{figure}

Para una señal cuadrada de frecuencia $\frac{f_{0_{1}}}{10} \approx 9,665Hz$ vemos que 
la respuesta oscila constantemente alrededor de cero, y dado que es un filtro pasabanda no reacciona a ninguno de los flancos (los flancos tendrían una frecuencia
prácticamente infinita y el filtro pasabanda, como vimos antes, no reacciona a dicha frecuencia). Comparando gráficos vemos que la respuesta a esta señal cuadrada se asemeja a la respuesta al escalón con la diferencia de que, debido a que estamos excitandola constantemente y no dejamos que alcance frecuencia cero, la respuesta nunca se extingue sino que repite constantemente el final de la respuesta al escalón.

\bigskip
\begin{figure}[h!]
\includegraphics[scale=1]{RtaCuadradaWo12.png}
\caption{Respuesta a la cuadrada de frecuencia $f_{0_{1}}$}
\end{figure}

Para la señal cuadrada de frecuencia $f_{0_{1}} \approx 96,65Hz$ tenemos que la 
respuesta, una vez estabilizada, es una senoidal de amplitud igual a la mitad de la 
de la señal cuadrada ($0,5V$ en este caso). Esto es lógico si vemos que la señal 
cuadrada generada se puede pensar como una continua de $0,5V$ más una cuadrada centrada en cero de $0,5V$ de amplitud. Dado que la frecuencia de la cuadrada esta en el rango de frecuencias entre los $W_{0_{i}}$ (más especificamente, coincide con el extremo $W_{0_{1}}$) entonces  la señal no se ve atenuada significativamente por el filtro. Por otro lado, el tiempo que tarda en estabilizarse la señal está relacionado con el tiempo que tarda el filtro en estabilizar la respuesta a la continua.


\newpage
\begin{figure}[h!]
\includegraphics[scale=1]{RtaCuadradaWo13.png}
\caption{Respuesta a la cuadrada de frecuencia $f_{0_{1}} \cdot 10$}
\end{figure}

Para la señal cuadrada de frecuencia $10 \cdot f_{0_{1}} \approx 966,5Hz$ tenemos que la respuesta oscila alrededor de los $0V$ cercanamente, coincidiendo con el comportamiento esperado del pasabanda para frecuencias mayores a las de la banda que deja pasar el filtro (que recordemos que va de $f_{0_{1}}$ a $f_{0_{2}}$, y debido a que estas frecuencias son cercanas entre sí entonces diez veces la primera ya se aleja considerablemente de dicha banda).

\subsubsection*{Frecuencia $f_{0_{2}}$}

Dado que $W_{0_{2}} \approx 650,2\frac{r}{s}$ para este caso tendremos que $f_{0_{2}} = \frac{W_{0_{2}}}{2 \cdot \pi} \approx 103,5Hz$. Debido a que las frecuencias $f_{0_{i}}$ son tan cercanas entre sí tendremos que los siguientes gráficos serán extremadamente parecidos a los anteriores, por lo que todo lo explicado previamente aplica de igual manera a lo siguiente y no será repetido.

\begin{figure}[t!]
\includegraphics[scale=0.95]{RtaCuadradaWo21.png}
\caption{Respuesta a la cuadrada de frecuencia $\frac{f_{0_{2}}}{10}$}
\includegraphics[scale=0.95]{RtaCuadradaWo22.png}
\caption{Respuesta a la cuadrada de frecuencia $f_{0_{2}}$}
\end{figure}

\clearpage
\begin{figure}[h!]
\includegraphics[scale=1]{RtaCuadradaWo23.png}
\caption{Respuesta a la cuadrada de frecuencia $f_{0_{2}} \cdot 10$}
\end{figure}

\section*{Circuito propuesto}

Se propone un circuito compuesto por dos Multiple Feedback Band-pass Filter (MFBP) . La elección del MFBP por sobre otro tipo de filtros se debe a la 
facilidad que aporta diseñar el circuito con este tipo de filtro. La principal
desventaja de este filtro es que es complicado conseguir secciones de frecuencias y Q altos ($Q > 20$) debido a las limitaciones de la ganancia de lazo abierto del amplificador operacional integrado.\footnote{Ver https://www.analog.com/media/en/training-seminars/design-handbooks/Op-Amp-Applications/Sections5-5-to-5-8.pdf, página 5.70} Dado que nuestra transferencia se compone por valores de $Q$ bajos ($Q < 20$) no habrá problemas en utilizar este tipo de filtro.

\subsection*{Transferencia del MFPB}

A continuación se presenta un diagrama de un filtro Multiple Feedback Band-pass (MFBP).

\begin{figure}[h!]
\includegraphics[scale=0.9]{MFBP.png}
\caption{Filtro MFBP}
\end{figure}

Para este tipo de filtro se tiene una transferencia del tipo
\[H_{MFBP}(s) = -H_{0} \cdot \frac{\frac{W_{0}}{Q} \cdot s}{s^2 + \frac{W_{0}}{Q} \cdot s + W_{0}^2}\]

Para simplificar las expresiones y cuentas tomaremos $C1 = C2 = C$. Resolviendo el circuito mediante el método de nodos obtenemos que la función de transferencia puede expresarse como

\bigskip
\[H_{MFBP}(s) = -\frac{R_{2}}{2 \cdot R_{1}} \cdot \frac{\frac{2}{C \cdot R_{2}} \cdot s}{s^2 + \frac{2}{C \cdot R_{2}} \cdot s + \frac{1+\frac{R_{1}}{R_{3}}}{C^2 \cdot R_{1} \cdot R_{2}}}\]

De las anteriores dos expresiones podemos inferir que
\[H_{0} = \frac{R_{2}}{2 \cdot R_{1}}\]
\[\frac{W_{0}}{Q} = \frac{2}{C \cdot R_{2}}\]
\[W_{0}^2 = \frac{1+\frac{R_{1}}{R_{3}}}{C^2 \cdot R_{1} \cdot R_{2}}\]

Despejando las $R_{i}$ en función de las demas variables obtenemos
\[R_{2} = \frac{2 \cdot Q}{C \cdot W_{0}}\]
\[R_{1} = \frac{R_{2}}{2 \cdot H_{0}}\]
\[R_{3} = \frac{R_{1}}{\frac{2 \cdot Q^2}{H_{0}} - 1}\]

donde $Q$ y $W_{0}$ son datos provenientes de nuestra función de transferencia, por lo que es cuestión tan solo de seleccionar un C adecuado, reemplazar y obtener los $R_{i}$. Antes de proceder a calcular los valores de estos elementos debemos expresar 
la función de transferencia $H(s)$ de forma tal que podamos aplicarle las ecuaciones previas. 

\bigskip
Previamente teníamos que
\[H(s) \approx \frac{3948 \cdot s^2}{(s^2+42,81 \cdot s + 607,3^2)
\cdot (s^2+46,05 \cdot s + 650,2^2)}\]

Reescribiendolo para que se adapte a la forma de los filtros MFBP\\ tendríamos
\[H(s) \approx 2 \cdot \frac{42,81 \cdot s}{(s^2+42,81 \cdot s + 607,3^2)}
\cdot \frac{46,05 \cdot s}{(s^2+46,05 \cdot s + 650,2^2)}\]

de forma tal que tenemos dos filtros MFBP, por lo tanto tendremos que encontrar las constantes $H_{0}$, $C$, $R_{1}$, $R_{2}$ y $R_{3}$ para cada uno. 

\section*{Cálculo de los componentes}

Comenzaremos calculando los valores para el MFBP de frecuencia \\$W_{0} \approx 607,3 \frac{r}{s}$ y $Q \approx 14,19$. La idea es conseguir valores ideales de resistencias que se aproximen lo más posible a valores normalizados (de forma tal de disminuir el error al normalizar) manteniendonos en un rango de resistencias de $1K\Omega$ a $1M\Omega$.
Para esto eligiremos el valor del capacitor C (normalizado) que de mejores resultados; dicho capacitor debe tener un valor de entre $1nF$ y $1 \mu F$.

\bigskip
El otro factor a variar junto con el C es el $H_{0}$ del MFBP. Tenemos que para la transferencia $H(s)$ debe suceder que $H_{0} \approx 2$, para lo cual debemos lograr que la multiplicación entre los $H_{0}$ de los filtros MFBP den aproximadamente 2. Si bien hay infinitas combinaciones posibles en la teoría, en la práctica nos vemos limitados por las otras condiciones del sistema. Se encontró que los mejores valores posibles (los valores que disminuyen la diferencia de las resistencias ideales con las normalizadas y cumplen las condiciones de rango de las resistencias) son
\[H_{0_{1}} = 0,5\]
\[H_{0_{2}} = 4\]

Volviendo al análisis del primer filtro, se llegó a que el mejor valor posible de C es
\[C = 47nF\]

obteniendose las siguientes resistencias
\[R_{1} \approx 994,0497621K\Omega\]
\[R_{2} \approx 994,0497621K\Omega\]
\[R_{3} \approx 1,236504032K\Omega\]

Normalizando obtenemos entonces que
\[R_{1} = 1M\Omega\]
\[R_{2} = 1M\Omega\]
\[R_{3} = 1,2K\Omega\]

Para el segundo filtro se llegó a un C tal que
\[C = 56nF\]

obteniendose las resistencias
\[R_{1} \approx 775,5193028K\Omega\]
\[R_{2} \approx 96,93991285K\Omega\]
\[R_{3} \approx 982,4374972\Omega\]

Normalizando obtenemos
\[R_{1} = 820K\Omega\]
\[R_{2} = 100K\Omega\]
\[R_{3} = 1K\Omega\]

\end{document}