\documentclass[11pt,a4paper]{report}
\usepackage[document]{ragged2e}
\usepackage{setspace}
\usepackage[utf8]{inputenc}
\usepackage{amsmath}
\usepackage{amsfonts}
\usepackage{amssymb}
\author{Marcos}
\begin{document}
\spacing{1.5}

\section*{Análisis general de la función de transferencia}

\subsection*{Tipo de filtro}

Dada la función de transferencia asignada
\[H(s)=\frac{3948 \cdot s^2}{s^4+88,86 \cdot s^3+7,935 \cdot 10^5 \cdot s^2+3,508 \cdot 10^7 \cdot s+1,559 \cdot 10^{11}}\]
procederemos a analizar su comportamiento en $s=0$ y $s\longrightarrow\infty$.

\bigskip
En primer lugar analizamos el caso $s=0$ para el cual tenemos que 
\[H(0) = \frac{3948 \cdot 0^2}{0^4+88,86 \cdot 0^3+7,935 \cdot 10^5 \cdot 0^2+3,508 \cdot 10^7 \cdot 0+1,559 \cdot 10^{11}}\]

de donde se deduce que
\[H(0) = 0\]

Analizando ahora el caso $s\longrightarrow\infty$ tenemos que
\[\lim_{s \to \infty} H(s) = 0\]
dado que el grado del denominador es dos veces mayor al del numerador.

\bigskip
En base a los valores obtenidos podemos entonces afirmar que se trata de un
filtro pasabanda dado que la transferencia es nula para frecuencias bajas y altas.

\subsection*{Ceros y polos}

En principio, es trivial ver que el único cero de la función $H(s)$ es
$s = 0$ y dado que está elevado al cuadrado se deduce que el cero es doble.

\bigskip
Por otro lado tenemos los polos, cuyo cálculo no es trivial. Dado que el denominador de la función es de grado cuatro tendremos entonces cuatro raíces. Debido a esto
y a que los coeficientes del polinomio complejizan el desarrollo del cálculo de 
dichas raíces, se calcularán entonces mediante calculadora. 

\bigskip
Obtenemos que
\[p_{1,2} \approx -21,40 \pm 606,9j\]
\[p_{3,4} \approx -23,03 \pm 649,8j\]
donde $p_{1,2}$ y $p_{3,4}$ son los pares conjuados que componen los cuatros
polos de $H(s)$, y donde el primer subíndice corresponde al conjugado cuya parte imaginaria es positiva mientras que el segundo subíndice corresponde al de parte imaginaria negativa.

\subsection*{Cálculo de $W_{0}$ y Q}

Dado que el denominador se compone de dos pares de raíces complejas conjugadas tendrá entonces un $W_{0}$ y Q para cada par. Para obtenerlos debemos primero reescribir la función $H(s)$ a la forma
\[H(s)=\frac{3948 \cdot s^2}{(a \cdot s^2+b \cdot s+c) \cdot (d \cdot s^2+e \cdot s+f)}\]
donde las letras corresponden a valores que debemos calcular de forma tal que dicha
ecuación sea equivalente a la fórmula original de $H(s)$.

\bigskip
Para conseguir esto es conveniente utilizar las raíces del denominador de $H(s)$ (los polos de la función) expresando la función como
\[H(s)=\frac{3948 \cdot s^2}{k \cdot (s+p_{1}) \cdot (s+p_{2}) \cdot (s+p_{3}) \cdot (s+p_{4})}\]
donde k es el factor del término $k*s^4$ y, para nuestra función en particular,
se da que $k=1$. 

\bigskip
Luego, multiplicando los polos conjuados entre sí obtenemos
\[H(s)=\frac{3948 \cdot s^2}{(s^2+(p1+p2) \cdot s+p_{1} \cdot p_{2}) \cdot (s^2+(p3+p4) \cdot s+p_{3} \cdot p_{4})}\]

\bigskip
Sean $z_{1}$ y $z_{2}$ dos números complejos conjugados tenemos que 

\[z1+z2=2\Re(z_{1}) = 2\Re(z_{2})\] 
\[z1 \cdot z2=|z_{1}|^2=|z_{2}|^2\] 

Aplicando estas propiedades a $p_{1,2}$ y 
$p_{3,4}$ obtenemos la expresión
\[H(s)=\frac{3948 \cdot s^2}{(s^2+2\Re(p1) \cdot s+|p_{1}|^2) \cdot (s^2+2\Re(p3) \cdot s+|p_{3}|^2)}\]

\bigskip
$W_{0}$ y Q vienen dados por $s^2+\frac{W_{0}}{Q} \cdot s + W_{0}^2$. Ejemplificando para el primer factor del denominador tendríamos entonces que 
\[W_{0_{1}} = |p1|\]
\[Q_{1} = \frac{W_{0_{1}}}{2\Re(p_{1})}\]


\end{document}