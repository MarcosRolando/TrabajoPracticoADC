\documentclass[11pt,a4paper]{report}
\usepackage[document]{ragged2e}
\usepackage{setspace}
\usepackage[utf8]{inputenc}
\usepackage{amsmath}
\usepackage{amsfonts}
\usepackage{amssymb}
\author{Marcos}
\begin{document}
\spacing{1.5}

\section*{Análisis general de la función de transferencia}

\subsection*{Tipo de filtro}

Dada la función de transferencia asignada
\[H(s)=\frac{3948*s^2}{s^4+88,86*s^3+7,935*10^5*s^2+3,508*10^7*s+1,559*10^{11}}\]
procederemos a analizar su comportamiento en $s=0$ y $s\longrightarrow\infty$.

\bigskip
En primer lugar analizamos el caso $s=0$ para el cual tenemos que 
\[H(0) = \frac{3948*0^2}{0^4+88,86*0^3+7,935*10^5*0^2+3,508*10^7*0+1,559*10^{11}}\]

de donde se deduce que
\[H(0) = 0\]

Analizando ahora el caso $s\longrightarrow\infty$ tenemos que
\[\lim_{s \to \infty} H(s) = 0\]
dado que el grado del denominador es dos veces mayor al del numerador.

\bigskip
En base a los valores obtenidos podemos entonces afirmar que se trata de un
filtro pasabanda dado que la transferencia es nula para frecuencias bajas y altas.

\subsection*{Ceros y polos}

En principio, es trivial ver que el único cero de la función $H(s)$ es
$s = 0$ y dado que está elevado al cuadrado se deduce que el cero es doble.

\bigskip
Por otro lado tenemos los polos, cuyo cálculo no es trivial. Dado que el denominador de la función es de grado cuatro tendremos entonces cuatro raíces. Debido a esto
y a que los coeficientes del polinomio complejizan el desarrollo del cálculo de 
dichas raíces, se calcularán entonces mediante calculadora. 

\bigskip
Obtenemos que
\[s_{1,2} \approx -21,40 \pm 606,9j\]


\end{document}