\documentclass[11pt,a4paper]{report}
\usepackage[document]{ragged2e}
\usepackage{setspace}
\usepackage[utf8]{inputenc}
\usepackage{amsmath}
\usepackage{amsfonts}
\usepackage{amssymb}
\author{Marcos}
\begin{document}
\spacing{1.5}

\section*{Análisis general de la función de transferencia}

Dada la función de transferencia asignada
\[H(s)=\frac{3948*s^2}{s^4+88.86*s^3+7.935*10^5*s^2+3.508*10^7*s+1.559*10^{11}}\]
procederemos a analizar su comportamiento en $s=0$ y $s\longrightarrow\infty$.

\bigskip
En primer lugar analizamos el caso $s=0$ para el cual tenemos que 
\[H(0) = \frac{3948*0^2}{0^4+88.86*0^3+7.935*10^5*0^2+3.508*10^7*0+1.559*10^{11}}\]

de donde se deduce que
\[H(0) = 0\]

Analizando ahora el caso $s\longrightarrow\infty$ tenemos que
\[\lim_{s \to \infty} H(s) = 0\]
dado que el grado del denominador es dos veces mayor al del numerador.

\bigskip
En base a los valores obtenidos podemos entonces afirmar que se trata de un
filtro pasabanda dado que la transferencia es nula para frecuencias bajas y altas.


\end{document}